% Options for packages loaded elsewhere
\PassOptionsToPackage{unicode}{hyperref}
\PassOptionsToPackage{hyphens}{url}
\PassOptionsToPackage{dvipsnames,svgnames*,x11names*}{xcolor}
%
\documentclass[
  english,
  man, noextraspace,floatsintext]{apa6}
\usepackage{lmodern}
\usepackage{amsmath}
\usepackage{ifxetex,ifluatex}
\ifnum 0\ifxetex 1\fi\ifluatex 1\fi=0 % if pdftex
  \usepackage[T1]{fontenc}
  \usepackage[utf8]{inputenc}
  \usepackage{textcomp} % provide euro and other symbols
  \usepackage{amssymb}
\else % if luatex or xetex
  \usepackage{unicode-math}
  \defaultfontfeatures{Scale=MatchLowercase}
  \defaultfontfeatures[\rmfamily]{Ligatures=TeX,Scale=1}
\fi
% Use upquote if available, for straight quotes in verbatim environments
\IfFileExists{upquote.sty}{\usepackage{upquote}}{}
\IfFileExists{microtype.sty}{% use microtype if available
  \usepackage[]{microtype}
  \UseMicrotypeSet[protrusion]{basicmath} % disable protrusion for tt fonts
}{}
\makeatletter
\@ifundefined{KOMAClassName}{% if non-KOMA class
  \IfFileExists{parskip.sty}{%
    \usepackage{parskip}
  }{% else
    \setlength{\parindent}{0pt}
    \setlength{\parskip}{6pt plus 2pt minus 1pt}}
}{% if KOMA class
  \KOMAoptions{parskip=half}}
\makeatother
\usepackage{xcolor}
\IfFileExists{xurl.sty}{\usepackage{xurl}}{} % add URL line breaks if available
\IfFileExists{bookmark.sty}{\usepackage{bookmark}}{\usepackage{hyperref}}
\hypersetup{
  pdftitle={Can Interracial Parasocial Interactions in Video Games Reduce Prejudice and Promote Support for Anti-Racism? A Study Proposal},
  pdfauthor={Naoyuki Sunami},
  pdflang={en-EN},
  pdfkeywords={video game, interracial contacts, parasocial relationships, prejudice},
  colorlinks=true,
  linkcolor=blue,
  filecolor=Maroon,
  citecolor=Blue,
  urlcolor=Blue,
  pdfcreator={LaTeX via pandoc}}
\urlstyle{same} % disable monospaced font for URLs
\usepackage{graphicx}
\makeatletter
\def\maxwidth{\ifdim\Gin@nat@width>\linewidth\linewidth\else\Gin@nat@width\fi}
\def\maxheight{\ifdim\Gin@nat@height>\textheight\textheight\else\Gin@nat@height\fi}
\makeatother
% Scale images if necessary, so that they will not overflow the page
% margins by default, and it is still possible to overwrite the defaults
% using explicit options in \includegraphics[width, height, ...]{}
\setkeys{Gin}{width=\maxwidth,height=\maxheight,keepaspectratio}
% Set default figure placement to htbp
\makeatletter
\def\fps@figure{htbp}
\makeatother
\setlength{\emergencystretch}{3em} % prevent overfull lines
\providecommand{\tightlist}{%
  \setlength{\itemsep}{0pt}\setlength{\parskip}{0pt}}
\setcounter{secnumdepth}{-\maxdimen} % remove section numbering
% Make \paragraph and \subparagraph free-standing
\ifx\paragraph\undefined\else
  \let\oldparagraph\paragraph
  \renewcommand{\paragraph}[1]{\oldparagraph{#1}\mbox{}}
\fi
\ifx\subparagraph\undefined\else
  \let\oldsubparagraph\subparagraph
  \renewcommand{\subparagraph}[1]{\oldsubparagraph{#1}\mbox{}}
\fi
% Manuscript styling
\usepackage{upgreek}
\captionsetup{font=singlespacing,justification=justified}

% Table formatting
\usepackage{longtable}
\usepackage{lscape}
% \usepackage[counterclockwise]{rotating}   % Landscape page setup for large tables
\usepackage{multirow}		% Table styling
\usepackage{tabularx}		% Control Column width
\usepackage[flushleft]{threeparttable}	% Allows for three part tables with a specified notes section
\usepackage{threeparttablex}            % Lets threeparttable work with longtable

% Create new environments so endfloat can handle them
% \newenvironment{ltable}
%   {\begin{landscape}\centering\begin{threeparttable}}
%   {\end{threeparttable}\end{landscape}}
\newenvironment{lltable}{\begin{landscape}\centering\begin{ThreePartTable}}{\end{ThreePartTable}\end{landscape}}

% Enables adjusting longtable caption width to table width
% Solution found at http://golatex.de/longtable-mit-caption-so-breit-wie-die-tabelle-t15767.html
\makeatletter
\newcommand\LastLTentrywidth{1em}
\newlength\longtablewidth
\setlength{\longtablewidth}{1in}
\newcommand{\getlongtablewidth}{\begingroup \ifcsname LT@\roman{LT@tables}\endcsname \global\longtablewidth=0pt \renewcommand{\LT@entry}[2]{\global\advance\longtablewidth by ##2\relax\gdef\LastLTentrywidth{##2}}\@nameuse{LT@\roman{LT@tables}} \fi \endgroup}

% \setlength{\parindent}{0.5in}
% \setlength{\parskip}{0pt plus 0pt minus 0pt}

% \usepackage{etoolbox}
\makeatletter
\patchcmd{\HyOrg@maketitle}
  {\section{\normalfont\normalsize\abstractname}}
  {\section*{\normalfont\normalsize\abstractname}}
  {}{\typeout{Failed to patch abstract.}}
\patchcmd{\HyOrg@maketitle}
  {\section{\protect\normalfont{\@title}}}
  {\section*{\protect\normalfont{\@title}}}
  {}{\typeout{Failed to patch title.}}
\makeatother
\shorttitle{VIDEO GAMES, INTERRACIAL INTERACTION, ANTI-RACISM}
\keywords{video game, interracial contacts, parasocial relationships, prejudice\newline\indent Word count: 1015}
\usepackage{csquotes}
\ifxetex
  % Load polyglossia as late as possible: uses bidi with RTL langages (e.g. Hebrew, Arabic)
  \usepackage{polyglossia}
  \setmainlanguage[]{english}
\else
  \usepackage[shorthands=off,main=english]{babel}
\fi
\ifluatex
  \usepackage{selnolig}  % disable illegal ligatures
\fi
\newlength{\cslhangindent}
\setlength{\cslhangindent}{1.5em}
\newlength{\csllabelwidth}
\setlength{\csllabelwidth}{3em}
\newenvironment{CSLReferences}[2] % #1 hanging-ident, #2 entry spacing
 {% don't indent paragraphs
  \setlength{\parindent}{0pt}
  % turn on hanging indent if param 1 is 1
  \ifodd #1 \everypar{\setlength{\hangindent}{\cslhangindent}}\ignorespaces\fi
  % set entry spacing
  \ifnum #2 > 0
  \setlength{\parskip}{#2\baselineskip}
  \fi
 }%
 {}
\usepackage{calc}
\newcommand{\CSLBlock}[1]{#1\hfill\break}
\newcommand{\CSLLeftMargin}[1]{\parbox[t]{\csllabelwidth}{#1}}
\newcommand{\CSLRightInline}[1]{\parbox[t]{\linewidth - \csllabelwidth}{#1}\break}
\newcommand{\CSLIndent}[1]{\hspace{\cslhangindent}#1}

\title{Can Interracial Parasocial Interactions in Video Games Reduce Prejudice and Promote Support for Anti-Racism? A Study Proposal}
\author{Naoyuki Sunami\textsuperscript{}}
\date{}


\affiliation{\phantom{0}}

\abstract{
Video games can be a powerful art form where players safely learn about racial out-groups---their stories, beliefs, values, and social norms without concerns of appearing as a racist. In video games, players can interact with out-group characters and form parasocial connections, which then theoretically could reduce out-group bias and increase support for antiracist social movements. However, other research suggested that White people may experience negative emotions in interracial interactions, which could then lead to disengagement and less support for anti-racism. In the proposed study, I will examine whether White people who interact with Black characters report lower prejudice towards Black people and higher support for anti-racism movements (e.g., Black Lives Matter and Kick Out Zwarte Piet) compared with those who play a video game with White characters only. I will also measure participants' heart rate variability as a correlate for emotion regulation while playing the video game. I expect that participants with higher (vs.~lower) heart rate variability while interacting with a Black character will report lower prejudice and higher support for anti-racism.
}



\begin{document}
\maketitle

Video games are vastly popular---an estimated 40\% of the worldwide population (70\% of the Dutch population) played video games in 2020 (IGN, 2020; Statista, 2021). Video games can be a powerful art medium where players can interact with racial outgroup characters, which could reduce prejudice and increase support for anti-racist social movements (Breves, 2020). If so, video games can provide a scalable intervention strategy to facilitate diversity, inclusion, and social justice. In the current study, I examine whether White players who play a video game with Black characters would report lower prejudice and higher support for anti-racism, compared with those who play a video game with only White characters.

\hypertarget{interracial-parasocial-interactions-in-video-games-and-implications-on-prejudice-and-anti-racism}{%
\subsection{Interracial Parasocial Interactions in Video Games and Implications on Prejudice and Anti-Racism}\label{interracial-parasocial-interactions-in-video-games-and-implications-on-prejudice-and-anti-racism}}

Many video games present characters to whom players can form one-sided, emotional relationships--called \emph{parasocial interactions} (Burgess \& Jones, 2020; Horton \& Wohl, 1956; Tyack \& Wyeth, 2017). These parasocial interactions can be interracial. For example, a White player playing \emph{Mafia III} can parasocially interact with Black characters and learn about their lives in the 1960s American south. These interracial contacts in video games are valuable since people often lack cross-race friends in real life (McDonald et al., 2013).

Do interracial parasocial interactions reduce prejudice and promote support for anti-racism? Studies on the parasocial contact hypothesis suggested that parasocial interactions with outgroups can reduce prejudice (Banas et al., 2020; Schiappa et al., 2005, 2006). In one study, participants played a quest from \emph{Skyrim} and interacted with a Black character in a virtual reality environment. After playing, participants reported lower prejudice against Black people compared with those who did not interact with a Black character (Banas et al., 2020). Lowered prejudice can facilitate more support for anti-racism: people with lower (vs.~higher) prejudice towards Black people endorsed the Black Lives Matter movement more (Ilchi \& Frank, 2021).~

\hypertarget{a-potential-downside-of-interracial-interactions-aversive-racism}{%
\subsection{A Potential Downside of Interracial Interactions: Aversive Racism}\label{a-potential-downside-of-interracial-interactions-aversive-racism}}

According to the aversive racism theory, interactions with racial out-groups may not necessarily reduce prejudice or improve support for anti-racism. The aversive racism theory suggests that White people tend to experience negative affect in interracial interactions, which then leads to disengagement and less helping behavior for outgroups (Gaertner, 1973; Gaertner \& Dovidio, 2005; Goff et al., 2008). In the context of video games, I can expect that some White players might disengage from interracial parasocial interactions to avoid experiencing negative emotions, resulting in no changes in prejudice or support for anti-racism.~

Some players may successfully regulate negative emotions and engage with interracial parasocial interactions. Indeed, Americans who effectively regulated their emotions reported lower prejudice towards Muslims following the Boston Marathon bombings than those who did not (Steele et al., 2019).~Thus, White players who successfully regulate their emotions in interracial interactions may report lower prejudice and higher support for anti-racism than those who do not. To measure the degree of emotion regulation, I will use the heart rate variability---variation in heart rate over time. People with higher (vs.~lower) heart rate variability regulated their emotions more effectively, indicating an adaptive response of the autonomic nervous system (Appelhans \& Luecken, 2006).

\hypertarget{current-study}{%
\subsection{Current Study}\label{current-study}}

I designed the current study to examine whether White people would report lower prejudice and higher support for anti-racism after interacting with a Black character in a video game. I suggest the following hypotheses. First, I hypothesize that White players who interact with a Black character will report lower prejudice (Hypothesis 1a) and higher support for anti-racism (Hypothesis 2a) than those who interact only with a White character. Following the aversive racism theory, I suggest an alternative hypothesis predicting null results---the levels of interracial parasocial interaction do not affect prejudice (Hypothesis 1b) or the support for anti-racism (Hypothesis 2b). Finally, I hypothesize that among White players interacting with a Black character, those with higher (vs.~lower) heart rate variability would report less prejudice (Hypothesis 3) and more support for anti-racism (Hypothesis 4).

\hypertarget{method}{%
\section{Method}\label{method}}

\hypertarget{sample-size-rationale-and-participants}{%
\subsection{Sample Size Rationale and Participants}\label{sample-size-rationale-and-participants}}

Since no reliable effect size is available for the current study, I draw from the median effect size (\emph{d} = 0.36) derived from 18,000+ effect sizes in social psychology (Lovakov \& Agadullina, 2021). I will also treat this effect size as the smallest effect size of interest to test the alternative hypotheses predicting null (Lakens, 2017). To achieve the 90\% power, and 5\% alpha, I will recruit 492 participants to detect \emph{d} = 0.36 between groups (164 x 3 groups).

\hypertarget{materials-procedure-and-measures}{%
\subsection{Materials, Procedure, and Measures}\label{materials-procedure-and-measures}}

Participants will come to the lab, and then wear a heart rate monitor for the heart rate variability recording. Participants will complete the demographics and a breathing task for the baseline heart rate variability (Time 1). Then, the participants will play a modded quest from Skyrim adopted from a previous study (Banas et al., 2020). I will randomly assign participants to one of the three conditions. In the Interracial Interaction Condition, participants will interact with Black characters to complete the quest. In the No Interracial Interaction Condition, participants will interact with White characters. In the Control Condition, participants will play \emph{Portal} for the same amount of time. I choose \emph{Portal} for the control condition since the game shares the first-person perspective as \emph{Skyrim} while not presenting parasocial interaction targets. During the gameplay, I will record participants' heart rate variability (Time 2).

After playing the video game, participants will complete a prejudice measure (Pettigrew \& Meertens, 1995, e.g., {``Black people have jobs that the Dutch people should have''}), and an ad-hoc scale measuring support for antiracism (asking participants how much they support or oppose ``Black Lives Matter,'' ``Kick Out Zwarte Piet,'' and ``Anti-Racism''). Finally, participants will complete a breathing task for the heart rate variability recording (Time 3).

\hypertarget{analytic-strategy}{%
\subsection{Analytic Strategy}\label{analytic-strategy}}

I will perform a One-Way ANOVA to examine if participants report different levels of prejudice (Hypotheses 1a and 1b) and support for anti-racism (Hypotheses 2a and 2b) across the conditions. To examine the moderating effect of heart rate variability across time, I will construct random-slope models predicting prejudice (Hypothesis 3) and anti-racism scores (Hypothesis 4) with the following predictors: Condition (categorical), Time, and Condition x Time interaction.

\newpage

\hypertarget{references}{%
\section{References}\label{references}}

\setlength{\parindent}{-0.5in}
\setlength{\leftskip}{0.5in}

\hypertarget{refs}{}
\begin{CSLReferences}{1}{0}
\leavevmode\hypertarget{ref-appelhansHeartRateVariability2006}{}%
Appelhans, B. M., \& Luecken, L. J. (2006). Heart rate variability as an index of regulated emotional responding. \emph{Review of General Psychology}, \emph{10}(3), 229--240. \url{https://doi.org/10.1037/1089-2680.10.3.229}

\leavevmode\hypertarget{ref-banasMetaAnalysisMediatedContact2020}{}%
Banas, J. A., Bessarabova, E., \& Massey, Z. B. (2020). Meta-{Analysis} on {Mediated Contact} and {Prejudice}. \emph{Human Communication Research}, \emph{46}(2-3), 120--160. \url{https://doi.org/10.1093/hcr/hqaa004}

\leavevmode\hypertarget{ref-brevesReducingOutgroupBias2020}{}%
Breves, P. (2020). Reducing {Outgroup Bias} through {Intergroup Contact} with {Non}-{Playable Video Game Characters} in {VR}. \emph{PRESENCE: Virtual and Augmented Reality}, \emph{27}(3), 257--273. \url{https://doi.org/10.1162/pres_a_00330}

\leavevmode\hypertarget{ref-burgessHarbourStrongFeelings2020}{}%
Burgess, J., \& Jones, C. (2020). {"}{I Harbour Strong Feelings} for {Tali Despite Her Being} a {Fictional Character}{"}: {Investigating Videogame Players}' {Emotional Attachments} to {Non}-{Player Characters}. \emph{Game Studies}, \emph{20}(1). \url{http://gamestudies.org/2001/articles/burgessjones}

\leavevmode\hypertarget{ref-gaertnerHelpingBehaviorRacial1973}{}%
Gaertner, S. L. (1973). Helping behavior and racial discrimination among liberals and conservatives. \emph{Journal of Personality and Social Psychology}, \emph{25}(3), 335--341. \url{https://doi.org/10.1037/h0034221}

\leavevmode\hypertarget{ref-gaertnerUnderstandingAddressingContemporary2005}{}%
Gaertner, S. L., \& Dovidio, J. F. (2005). Understanding and {Addressing Contemporary Racism}: {From Aversive Racism} to the {Common Ingroup Identity Model}. \emph{Journal of Social Issues}, \emph{61}(3), 615--639. \url{https://doi.org/10.1111/j.1540-4560.2005.00424.x}

\leavevmode\hypertarget{ref-goffSpaceUsStereotype2008}{}%
Goff, P. A., Steele, C. M., \& Davies, P. G. (2008). The space between us: {Stereotype} threat and distance in interracial contexts. \emph{Journal of Personality and Social Psychology}, \emph{94}(1), 91--107. \url{https://doi.org/10.1037/0022-3514.94.1.91}

\leavevmode\hypertarget{ref-hortonMassCommunicationParaSocial1956}{}%
Horton, D., \& Wohl, R. R. (1956). Mass {Communication} and {Para}-{Social Interaction}. \emph{Psychiatry}, \emph{19}(3), 215--229. \url{https://doi.org/10.1080/00332747.1956.11023049}

\leavevmode\hypertarget{ref-ignThreeBillionPeople2020}{}%
IGN. (2020, August 14). \emph{Three {Billion People Worldwide Now Play Video Games}, {New Report Shows} - {IGN}}. \url{https://www.ign.com/articles/three-billion-people-worldwide-now-play-video-games-new-report-shows}

\leavevmode\hypertarget{ref-ilchiSupportingMessageNot2021}{}%
Ilchi, O. S., \& Frank, J. (2021). Supporting the {Message}, {Not} the {Messenger}: {The Correlates} of {Attitudes} towards {Black Lives Matter}. \emph{American Journal of Criminal Justice}, \emph{46}(2), 377--398. \url{https://doi.org/10.1007/s12103-020-09561-1}

\leavevmode\hypertarget{ref-lakensEquivalenceTestsPractical2017}{}%
Lakens, D. (2017). Equivalence {Tests}: {A Practical Primer} for t {Tests}, {Correlations}, and {Meta}-{Analyses}. \emph{Social Psychological and Personality Science}, \emph{8}(4), 355--362. \url{https://doi.org/10.1177/1948550617697177}

\leavevmode\hypertarget{ref-lovakovEmpiricallyDerivedGuidelines2021}{}%
Lovakov, A., \& Agadullina, E. R. (2021). Empirically derived guidelines for effect size interpretation in social psychology. \emph{European Journal of Social Psychology}, \emph{n/a}(n/a). \url{https://doi.org/10.1002/ejsp.2752}

\leavevmode\hypertarget{ref-mcdonaldContributionsRacialSociobehavioral2013}{}%
McDonald, K. L., Dashiell-Aje, E., Menzer, M. M., Rubin, K. H., Oh, W., \& Bowker, J. C. (2013). Contributions of {Racial} and {Sociobehavioral Homophily} to {Friendship Stability} and {Quality Among Same}-{Race} and {Cross}-{Race Friends}. \emph{The Journal of Early Adolescence}, \emph{33}(7), 897--919. \url{https://doi.org/10.1177/0272431612472259}

\leavevmode\hypertarget{ref-pettigrewSubtleBlatantPrejudice1995}{}%
Pettigrew, T. F., \& Meertens, R. W. (1995). Subtle and blatant prejudice in western {Europe}. \emph{European Journal of Social Psychology}, \emph{25}(1), 57--75. \url{https://doi.org/10.1002/ejsp.2420250106}

\leavevmode\hypertarget{ref-schiappaParasocialContactHypothesis2005}{}%
Schiappa, E., Gregg, P. B., \& Hewes, D. E. (2005). The {Parasocial Contact Hypothesis}. \emph{Communication Monographs}, \emph{72}(1), 92--115. \url{https://doi.org/10.1080/0363775052000342544}

\leavevmode\hypertarget{ref-schiappaCanOneTV2006}{}%
Schiappa, E., Gregg, P. B., \& Hewes, D. E. (2006). Can one {TV} show make a difference? {A Will} \& {Grace} and the parasocial contact hypothesis. \emph{Journal of Homosexuality}, \emph{51}(4), 15--37. \url{https://doi.org/10.1300/J082v51n04_02}

\leavevmode\hypertarget{ref-statistaHoursSpentPlaying2021}{}%
Statista. (2021, April). \emph{Hours spent on playing video games per week in the {Netherlands} 2020}. {Statista}. \url{https://www-statista-com.udel.idm.oclc.org/forecasts/1226826/hours-spent-on-playing-video-games-per-week-in-the-netherlands}

\leavevmode\hypertarget{ref-steeleEmotionRegulationPrejudice2019}{}%
Steele, R. R., Rovenpor, D. R., Lickel, B., \& Denson, T. F. (2019). Emotion regulation and prejudice reduction following acute terrorist events: {The} impact of reflection before and after the {Boston Marathon} bombings. \emph{Group Processes \& Intergroup Relations}, \emph{22}(1), 43--56. \url{https://doi.org/10.1177/1368430217706182}

\leavevmode\hypertarget{ref-tyackExploringRelatednessSingleplayer2017}{}%
Tyack, A., \& Wyeth, P. (2017). Exploring relatedness in single-player video game play. \emph{Proceedings of the 29th {Australian Conference} on {Computer}-{Human Interaction}}, 422--427. \url{https://doi.org/10.1145/3152771.3156149}

\end{CSLReferences}


\end{document}
